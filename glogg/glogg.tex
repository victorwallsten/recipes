\documentclass{book}

\usepackage[nonumber]{cuisine}

\begin{document}

\thispagestyle{empty}

\begin{recipe}{Christines glögg}{}{}
  \ing[5]{st}{Potatis, fast}
  \ing[25]{g}{Ingefära, hel}
  Skiva potatis och ingefära\newline grovt.
  \ing[5]{l}{Svagdricka}
  \ing[2,5]{kg}{Strösocker}
  \ing[250]{g}{Russin}
  \ing[30]{g}{Nejlika, hel}
  Blanda all potatis, ingefära, svagdricka, strösocker, russin och nejlika i
  ett stort kärl (tio liter är lagom). Se till att sockret löses upp.
  \ing[50]{g}{Jäst, färsk}
  Tillsätt smulad jäst och rör om.
  \newstep
  Låt stå i rumstemperatur tills jäsningen är färdig (vanligt-vis efter 3--4
  veckor), gärna med lock på glänt eller mot-svarande.
  \newstep
  Sila över drycken till ett annat kärl.
  \newstep
  Låt stå i en vecka för att klarna.
  \newstep
  Buteljera. Försök att undvika bottensatsen, exempelvis genom att använda en
  hävert eller försiktigt sleva.

\end{recipe}

\end{document}
